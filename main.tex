\documentclass{article}
\usepackage{graphicx}
\usepackage{cite}
\usepackage{xeCJK}
\setCJKmainfont{SimSun} % 宋体,支持中文显示

\title{云编译测试文档}
\author{作者姓名}
\date{\today}

% 推荐使用 Noto Sans CJK SC 或 WenQuanYi Micro Hei,避免 SimSun 字体缺失
\setCJKmainfont{Noto Sans CJK SC}
% 如编译仍报错,可改为:
% \setCJKmainfont{WenQuanYi Micro Hei}
% 或 \setCJKmainfont{AR PL UMing CN}

\begin{document}

\maketitle

\section{引言}
这是一个用于测试云编译的简单 LaTeX 文档。

本项目旨在展示如何在不同环境下编译和预览 PDF 文档,尤其是支持中文内容的排版。通过集成 xeCJK 包和设置中文字体,可以确保文档在各种平台上正确显示中文字符。

下面将展示图片插入、参考文献等常用功能,并扩展部分中文示例文本以便测试字体兼容性。

\section{图片示例}
\begin{figure}[h]
    \centering
    \includegraphics[width=0.5\textwidth]{figures/figure1.png}
    \caption{示例图片}
    \label{fig:example}
\end{figure}

\section{参考文献}
如文献~\cite{testref}所示。

\bibliographystyle{plain}
\bibliography{refs}

\section{中文示例段落}
在这个段落中,我们可以看到中文字体的显示效果。无论是在 Windows 还是 macOS 系统,只要安装了宋体(SimSun),都可以正常显示中文内容。

此外,LaTeX 还支持多种中文字体,例如仿宋、黑体、楷体等,可以通过 \texttt{\textbackslash setCJKmainfont} 命令进行切换。

\end{document}